\documentclass[a4paper]{article}

\usepackage[utf8]{inputenc}
\usepackage[T1]{fontenc}
\usepackage[russian]{babel}
\usepackage{amsmath,amssymb,amsthm}
\usepackage{mathrsfs}
\usepackage[matrix,arrow,curve]{xy}
\usepackage[left=2cm,right=2cm,top=2cm,bottom=2cm,bindingoffset=0cm]{geometry}

\usepackage{multicol}
\setlength{\columnsep}{0.5cm}

\usepackage{enumitem}

\renewcommand{\labelenumii}{\arabic{enumii})}
\renewcommand{\labelenumiii}{\roman{enumiii})}


\begin{document}

\noindent\textbf{Поиск экстремумов функции.}

\begin{enumerate}
	\item (7) Можно ли выбрать несчетное число попарно не пересекающихся интервалов на прямой? А кругов на плоскости (точка не считается за круг)?\\
Нет. Каждый отрезок содержит рациональные числа. Для каждого отрезка его рациональные числа уникальны. Рациональные числа одного отрезка больше всех рациональных чисел другого, если тот лежит левее. А значит мы можем сопоставить каждый отрезок любому рациональному числу на нем, тем самым сделав инъекцию в $\mathbb Q$, а значит и в $\mathbb N$. Поскольку $\mathbb Q^2$ тоже счетно, то это же рассуждение применимо и для кругов, а значит их множество тоже не может быть счетно. 
	\item (7) Число $a\in \mathbb R$ называется алгебраическим, если найдутся такие $q_1,\dots,q_n\in \mathbb Q$, что
	\[
	    a^n + q_1a^{n-1} + a_2q^{n-2} + \dots + q_{n-1}a + q_n = 0.
	\]
	Обозначим множество алгебраических чисел через $\mathcal{A}$. Докажите, что множество $\mathcal{A}$ счетно.
	
	\item Найдите предельные точки множества $A$, если
	\begin{enumerate}
	    \item (4) $A = \{\frac{(1 - (-1)^n)2^n + 1}{2^n + 3}\ \mid n\in \mathbb N\}$\\
    Решение:\\
        $A$ -- совокупность элементов двух последовательностей $x_n = \frac{2^{(2n + 1)+1} + 1}{2^{2n+1} + 3}, y_n = \frac{1}{2^{2n} + 3}$. $x_n \rightarrow \frac{2 + 0}{1 + 0} = 2, y_n \rightarrow 0$, $x_n \neq 2, y_n \neq 0$. Это значит, что в $A$ в любых проколотых окрестностях 2 и 0 лежит бесконечное число точек, а значит 2 и 0 -- предельные точки. А вне этих окрестностей лежит конечное число точек, а значит и предельных точек $A$ больше нет.
        \item (4) $A = \{\frac{(1 + \cos (\pi n)) \log (3n) + \log(n)}{\log (2n)}\ \mid n\in \mathbb N\}$\\
    Решение:\\
    Похоже на предыдущий пункт. $A$ -- совокупность элементов $x_n = \frac{\log(2n+1)}{\log(2(2n+1))}, y_n = \frac{2\log(3(2n)) + \log(2n)}{\log(2(2n))}, x_n \rightarrow 1, y_n \rightarrow 3$. По аналогии с предыдушим пунктом, 1 и 3 -- предельные точки и так же, как и там, $x_n$ и $y_n$ имеют только конечное число элементов вне окрестностей этих точек, а значит и другие точки -- не предельные.
	\end{enumerate}
	
	\item (5) Пусть $A\subset \mathbb R$ --- счетное множество. Найдите $(\mathbb R\smallsetminus A)'$.\\
Ответ: $\mathbb R$. Предположим противное, т.е. $\exists x \in \mathbb R, \mathcal{E} > 0$ такие, что $\forall y \in (\mathbb R\smallsetminus A)$ лежит вне проколотой $\mathcal{E}$ окрестности $x$. Тогда точки, которые лежали внутри этой окрестности принадлежат $A$, а это значит, что у $A$ есть несчетное подмножество, чего не может быть. 
	\item (7) Докажите, что $A''\subset A'$ для всякого $A\subset \mathbb R$.
\end{enumerate}



\end{document}
