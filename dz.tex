\documentclass[a4paper]{article}





\usepackage[utf8]{inputenc}
\usepackage[T1]{fontenc}
\usepackage[russian]{babel}
\usepackage{amsmath,amssymb,amsthm}
\usepackage{mathrsfs}
\usepackage[matrix,arrow,curve]{xy}
\usepackage[left=2cm,right=2cm,top=2cm,bottom=2cm,bindingoffset=0cm]{geometry}

\usepackage{multicol}
\setlength{\columnsep}{0.5cm}

\usepackage{enumitem}
%\renewcommand{\labelenumii}{\arabic{enumii}.}

\usepackage{import}


\newtheorem*{rem}{Remark}
\newtheorem{lemma}{Lemma}
\newtheorem{cor}{Corollary}

\def\Im{\mathrm{Im}\,}
\def\eps{\varepsilon}
\def\Int{\mathrm{Int}}
\def\Cl{\mathrm{Cl}}

\def\sh{\mathrm{sh}}
\def\ch{\mathrm{ch}}
\def\th{\mathrm{th}}

\def\arcsh{\mathrm{arcsh}}
\def\arcch{\mathrm{arcch}}
\def\arcth{\mathrm{arcth}}

\def\vphi{\varphi}

\def\pr{\partial}


\begin{document}

\newcommand\HeaderDZ[5]{
\begin{center}
		\textbf{Домашнее задание #3, #2. Дедлайн #4, #5.}\\
		Группа #1.Б10-пу.\\
\end{center}
\vspace{-\baselineskip}
\bigskip
\bigskip
}



\renewcommand{\labelenumii}{\arabic{enumii})}
\renewcommand{\labelenumiii}{\roman{enumiii})}

\HeaderDZ{19}{11.10.19}{4}{22.10.19}{14:00}

\begin{enumerate}
    \item (7) Пусть $0<b_1<a_1$ и $a_{n+1} = \frac{a_n + b_n}{2}$, $b_{n+1} = \sqrt{a_nb_n}$. Покажите, что $a_n$ и $b_n$ --- сходятся и $\lim\limits_{n\to+\infty}a_n = \lim\limits_{n\to+\infty}b_n$.\\
    Решение:\\
    $\frac{a_{n-1} + b_{n-1}}{2} \ge \sqrt{a_{n-1}b_{n-1}}$, так как $a_{n-1}$ и $b_{n-1}$ неотрицательные (неравенство Коши). Тогда $a_{n} \ge b_{n}$. Тогда $a_{n+1} = \frac{a_n + b_n}{2} \le \frac{a_n + a_n}{2} \le a_n$. Т.е. $a_n$ убывает. Также $a_n$ ограничена снизу нулем, поэтому она сходится.\\
    $b_{n+1} = \sqrt{a_nb_n} \ge \sqrt{b_nb_n} = b_n$. Т.е. $b_n$ возрастает. Но также $b_{n+1} = \sqrt{a_nb_n} \le \sqrt{a_na_n} = a_n$. Т.е. $b_{n+1} \le \sup \{a_n\}$. Т.е. $b_n$ ограничена сверху. Значит тоже сходится.\\
    Пусть $a_n \rightarrow a$, $b_n \rightarrow b$. Сделаем предельный переход. $a = \frac{a+b}{2} \Leftrightarrow a = b$.  
    \item (7) Пусть $x_n$ --- числовая последовательность. Предположим, что последовательности $x_{2n}, x_{2n+1}, x_{3n}$ --- сходятся. Докажите, что последовательность $x_n$ --- тоже сходится.\\
    Решение:\\
    Пусть $x_{2n} \rightarrow a, x_{3n} \rightarrow b$. Рассмотрим $x_{6n}$. $x_{6n}$ -- последовательность и $x_{3n}$ и $x_{2n}$. Если последовательность куда-то стремится, то туда же стремится и ее подпоследовательность а это значит, что предел подпоследовательности равен пределу последовательности. $x_{6n} \rightarrow a$ и одновременно $x_{6n} \rightarrow b$. Значит $a=b$.\\
    Возьмем $x_{3(2n+1)}$. По аналогии понимаем, что переделы $x_{3n}$ и $x_{2n+1}$ равны. А значит и пределы $x_{2n+1}$ и $x_{2n}$ тоже. А эти две последовательности составляют $x_n$. А значит и $x_n \rightarrow a$.
    \item (7) Пусть $x_n$ --- числовая последовательность, причем $\lim\limits_{n\to+\infty} (x_n - 2x_{n+1}) = 0$. Покажите, что $x_n$ сходится.\\
    Перепишем условие:\\
    $\forall \eps > 0\ \exists N\ \forall n > N\ |x_n - 2x_{n+1}| < \eps$\\
    Если мы взяли какое-то $\eps$ и первый $|x_n| \ge \eps$, то $$|x_n - 2x_{n+1}| < |x_n| \Leftrightarrow 
    (x_n - 2x_{n+1})^2 < (x_n)^2 \Leftrightarrow
    x_{n+1}(x_n - x_{n+1}) > 0
    $$
    Получается, что в этом случае $|x_n|$ все время уменьшается, а также $x_n$ стремится к 0, так как либо $x_n > x_{n+1} > 0$, либо $x_n < x_{n+1} < 0$.\\
    Если $|x_n| < \eps$, то\\
    $$
    |x_n - 2x_{n+1}| < \eps \Leftrightarrow
    \begin{cases}
        x_n - 2x_{n+1} < \eps\\
        x_n - 2x_{n+1} > -\eps
    \end{cases}
    \begin{cases}
        x_{n+1} > \frac{x_n - \eps}{2}\\
        \frac{x_n + \eps}{2} > x_{n+1}
    \end{cases} \Rightarrow
    \begin{cases}
        x_{n+1} > \frac{x_n - \eps}{2}\\
        \eps > x_{n+1}
    \end{cases} \Rightarrow |x_{n+1}| < \eps.
    $$
    Получается, что для любого $\eps > 0$, начиная с какого-то $N$, все члены $x_n$ $< \eps$ или стремятся к нулю, тогда $x_n$ стремится к нулю.
    \item Укажите, какие из следующих утверждений верны для любого множества $A\subset \mathbb R^2$ (докажите, или приведите контрпример):

    \begin{enumerate}
        \item (3) $\Int (A\cap B) = (\Int A)\cap(\Int B)$\\
        Если $a$ лежит в $\Int (A\cap B)$, то она лежит в пересечении $A$ и $B$ с каким-то шариком, а значит она лежит с этим шариком в $A$ и $B$, т.е. $\Int (A\cap B)\subset (\Int A)\cap (\Int B)$.\\
        Если $a$ лежит в $(\Int A)\cap(\Int B)$, то она лежит и в $\Int A$ и в $\Int B$. Т.е. лежит с каким-то шариком и в $A$ и в $B$. Возьмем шарик меньшего радиуса и тогда он будет лежать в пересечении $A$ и $B$, а значит и в $\Int(A\cap B)$.
        Есть включение в обе стороны, а значит и равенство.
        \item (3) $\Int (A\cup B) = (\Int A)\cup(\Int B)$\\
        Пускай $A$ -- одна половина круга с радиусом 1 и центром в (0, 0). А $B$ -- другая. Диаметр, по которому разделили, лежит в обоих множествах. Тогда $\Int$ обоих множеств будет половиной круга без этого диаметра. Тогда он не будет лежать и в объединении внутренностей. А в $\Int$ объединения этот диаметр будет лежать, потому что объединение -- весь круг.
        \item (3) $\Cl (A\cap B) = (\Cl A)\cap(\Cl B)$\\
        Возьмем опять круг с радиусом 1 и центром в (0,0). $A$ -- одна половина без разделяющего диаметра, $B$ -- другая. В замыкании $A$ и $B$ этот диаметр есть, поэтому он будет и в пересечении замыканий. Но их пересечение равно $\varnothing$, а значит и замыкание пересечения тоже $\varnothing$.
        \item (3) $\Cl (A\cup B) = (\Cl A)\cup(\Cl B)$\\
        В $\Cl A$ лежат точки $a$, для которых верно, что $\forall r > 0\ A\cap B(a, r) \neq \varnothing$. Понятно, что они все будут лежать в $\Cl (A\cup B)$. Для $B$ аналогично, поэтому $\Cl (A\cup B) \supset (\Cl A)\cup(\Cl B)$.\\
        А если $a$ в $\Cl(A\cup B)$, то $\forall r > 0\ (A\cup B)\cap B(a, r) \neq \varnothing$. А вот эти точечки в $A\cup B$, которые дают непустое пересечение, лежат хотя бы в одном из множеств. Тогда $a$ будет в $\Cl A$ или $\Cl B$.\\
        Получили включение в обе стороны, а значит и равенство.
        \item (3) $\Int (A\smallsetminus B) = \Int (A)\smallsetminus \Cl(B)$\\
        $a \in \Int (A\smallsetminus B)$ равносильно тому, что она лежит в $A$, не лежит в $B$ и у нее существует окрестность, которая целиком лежит в $A$ и не пересекается с $B$.\\
        Точки $\Int A$ лежат в $A$ и у них существует окрестность, которая целиком лежит в $A$. Точки $\Cl B$ -- это все точки, которые либо лежат в $B$, либо для любой их окрестности ее пересечение с $B$ не ноль.\\
        Получается, что $\mathbb R^2 \smallsetminus \Cl(B)$ -- точки, которые не лежат в $B$ и у которых есть окрестность, не пересекающаяся с $B$.\\
        $\Int (A)\smallsetminus \Cl(B) = \Int(A)\cap (\mathbb R^2 \smallsetminus \Cl(B)) = \Int(A\smallsetminus B)$
        \item (3) $\Cl (A\smallsetminus B) = \Cl(A)\smallsetminus \Int(B)$\\
        $A = \{(\frac{1}{n}, 1)\ \mid n\in \mathbb N\}$, $B = \{(\frac{1}{n}, 1)\ \mid n\in \mathbb N\smallsetminus\{1\}\}$\\
        $A\smallsetminus B = \{(1, 1)\},\ \Cl(A\smallsetminus B) = \varnothing$.\\
        $\Cl A = \{(0, 1)\},\ \Int B = \varnothing,\ \Cl A \smallsetminus \Int B = \{(0, 1)\} \neq \Cl(A\smallsetminus B)$
    \end{enumerate}
    
    \item В этом задании $A\subset \mathbb R^2$. Вычислите $\Int\,A$, $\Cl\,A$, $A'$ и $\partial A$, если
    \begin{enumerate}
        \item (4) $A = \{ (x,y)\ \mid\ x+y = 1,\ |x|<1 \}$\\
        $A = \{ (x,y)\ \mid\ y = 1-x,\ |x|<1 \} \Leftrightarrow A = \{ (x,1-x)\ \mid\ |x|<1 \}$. Это отрезок, поэтому в $\mathbb R^2$ $\Int A = \varnothing$. (Пересечение любой окрестности точки из $A$ с $A$ будет отрезком, а не кругом.)\\
        $A' = \{ (x,1-x)\ \mid\ |x|\le1 \}$, так как для всех точек $A$ очевидно, что они лежат в $A'$, а также у нас есть точки (1, 0) и (-1, 2), в любой окрестности которых есть точки $A$.\\
        Изолированных точек нет, поэтому $\Cl A = A'$\\
        $\partial A = \Cl A \smallsetminus \Int A = \Cl A$.
        \item (4) $A = \mathrm{\bf C}\times \mathrm{\bf C}$.\\
        Рассмотрим какое-то $(x, y) \in A$ и $\eps > 0$. Зафиксируем $x$. Поскольку $y \in \mathrm{\bf C}$, а Канторово множество нигде не плотно, то в $\eps$-окрестности $y$ существует $y' \notin \mathrm{\bf C}$. А значит $\eps$-окрестность $(x, y)$ полностью не лежит в $A$. Тогда $\Int A = \varnothing$.\\
        Так как $\mathrm{\bf C}' = \mathrm{\bf C}$, то и $A' = A$. И тогда из этого следует, что у $A$ нет изолированных точек.\\
        Тогда $\Cl A = A'$, $\partial A = A'$.
        \item (4) $A = \mathbb Q\times (\mathbb R\smallsetminus \mathbb Q)$\\
        $\Int A = \varnothing$, потому что в любой окрестности $a \in A$ есть точка с иррациональной первой координатой.\\
        $\mathbb R\smallsetminus \mathbb Q = \mathbb I$. Тогда $A$ -- множество точек с рациональной первой координатой и иррациональной второй. В любой окрестности любого вещественного числа есть и иррациональное и рациональное число, поэтому в любой окрестности $a \in \mathbb R^2$ будет $b \in A$. И тогда $A' = \mathbb R^2$.\\
        Тогда $\Cl A = \mathbb R^2$ и $\partial A = \mathbb R^2$.
        \item (4) $A = \{(x,y)\ \mid\ [x] \leq [y]\}$\\
        А состоит из точек, которые лежат слева от прямой $y=x$, а также точек квадратов, которые пересекает эта прямая, и углы которых -- целые точки.\\
        Тогда $\Cl A = A$, так как у любой точки из $A$ в любой окрестности лежит точка из $A$, а у любой не из $A$ можно взять окрестность с радиусом равным половине минимального расстояния до границы какого-нибудь из названных выше квадратов, и в этой окрестности не будет точек $A$.\\
        Тогда внутренностью $A$ будет $A$ без левых и нижних сторон этих квадратов, потому что у любой точки из $A$ можно взять окрестность с радиусом равным половине минимального расстояния до нижних или левых сторон этих квадратов, и в этой окрестности все точки будут из $A$.\\
        Тогда получается, граница равне левым и нижним сторонам этих квадратов, а $A' = \Cl A$.
    \end{enumerate}
    ($[\cdot]$ обозначает целую часть, а $\mathrm{\bf C}$ --- Канторово множество).
    
    \item (10) Пусть $A\subset \mathbb R^n$ --- открытое множество. Покажите, что найдется счетный набор открытых шаров $B_1,B_2,B_3,\dots$ такой, что
    \[
        A = \bigcup_{i\geq 1} B_i.
    \]\\
    $C\subset \mathbb R$, $C$ -- открыто. Если $C$ открыто, то $C$ -- объединение открытых лучей и интервалов. (Иначе у нас будут точки, в любых $\eps$-окрестностях которых не будет точек из $C$ хотя бы с одной из сторон).\\
    Интервал на прямой равен открытому шару с центром в середине интервала и радиусом в половину его длины.\\
    Открытый луч можно покрыть счетным числом шаров с центрами в целых числах и радиусом, большим 1. Надо только взять радиус меньше 1 у шара с центром в первом целом числе луча, чтобы не выйти за границу и у следующего шара радиус побольше, чтобы покрыть непокрытую часть.\\
    Таким образом мы можем из объединения счетного числа шаров получить открытое подмножество $\mathbb R$.\\
    Возьмем $C_1, C_2, \cdots C_n \subset \mathbb R$, такие что $C_i$ -- множество всех значений $i$-ой координаты точек из $A$. Тогда $C_1\times C_2\times \cdots\times C_n = A$. Просто по определению произведения. Докажем, что все $C$ открытые.\\
    Возьмем какую-то $a \in A$. Мы знаем, что $\exists \eps > 0$ такое,что все точки в $\eps$-окрестности $a$ лежат в $A$. Пусть $x_i$ -- $i$-ая координата $a$. Тогда в $\eps$-окрестности $a$ есть точки со всеми $i$-ыми координатами из $(x_i-\eps, x_i+\eps)$, а значит все они лежат в $C_i$. Тогда любой $x_i$ лежит в $C_i$ с какой-то окрестностей, а значит $C_i$ -- открытое.\\
    Тогда $A$ равно счетному произведению открытых множеств, каждое из которых равно объединению счетного количества шаров. И тогда $A$ можно составить из счетного количества шаров.
    
    \item (10) Пусть $A\subset \mathbb R^2$. Сколько различных множеств можно получить, применяя к $A$ операции $\Cl$ и $\Int$?\\
    В $\Cl A$ лежат все точки $A$, а также все предельные точки $A$. Так как $A'' \subset A'$, то в $\Cl A$ лежат и все предельные точки $\Cl A$. Тогда $\Cl(\Cl A) = \Cl A$, т.е. $\Cl$ друг на друга ставить не имеет смысла.\\
    Аналогично с $\Int A$. $\Int A$ -- открытое множество. Тогда $\Int(\Int A) = \Int A$.\\
    Рассмотрим $\Int(\Cl A)$. $\Int A \subset \Int(\Cl A)$, так как в $\Cl A$ есть все точки $A$.\\
    Если $a \in A$ -- изолированная в $A$, то она не может быть внутренней $\Cl A$, так как иначе у нас есть бесконечно близкие к $a$ точки, которые либо сами лежат в $A$, либо имеют бесконечно близко точку из $A$. И тогда $a$ не изолированная. Поэтому в $\Int(\Cl A)$ не будет изолированных точек $A$. Также $a$ не может лежать в окрестности внутренней точки $\Cl A$, потому что сама тогда будет внутренней. Из этого следует, что если убрать изолированные точки $A$ из $\Cl A$, то $\Int (\Cl A)$ не изменится.\\
    Тогда $\Int(\Cl A) = \Int (A') \cup \Int A$. Так как $\Int A\subset A'$, то $\Int(\Cl A) = \Int (A')$\\
    Рассмотрим $\Cl(\Int A)$.\\
    $\Cl(\Int A)$ -- изолированные и предельные точки $\Int A$. Т.е. просто $(\Int A)'$.\\
    $\Int(\Cl(\Int(\Cl A))) = \Int((\Int(A'))')$. Докажем, что $\Int((\Int(A'))') = \Int(A').$\\
    Если $a$ лежит в $\Int(A')$, то она лежит в $(\Int(A'))'$, а также та окрестность, вместе с которой она лежит в  $A'$ лежит в $(\Int(A'))'$, поэтому она лежит вместе с какой-то окрестностью в $(\Int(A'))'$, а значит и в $\Int((\Int(A'))')$\\
    Если $a$ лежит в $\Int((\Int(A'))')$, то она лежит вместе с какой-то окрестностью в $(\Int (A'))'$, каждая из точек этой окрестности имеет точку из $\Int (A')$ в любой окрестности, а каждая из точек $\Int(A')$ лежит в $A'$ с какой-то окрестностью. И тогда $a$ имеет точку из $A$ в любой окрестности, т.е. лежит в $A'$.\\
    Тогда все точки $\Int((\Int(A'))')$ лежат в $A'$, а значит и в $\Int(A')$. Доказали включение в обе стороны, а значит и равенство.\\
    Докажем, что $\Cl(\Int(\Cl(\Int A)))$ = $\Cl(\Int A)$, т.е. $(\Int ((\Int A)'))' = (\Int A)'$\\
    Так как $\Int A \subset (\Int A)'$, то $\Int A \subset \Int((\Int A)')$, так как мы берем внутренность большего множества. И тогда $(\Int A)' \subset (\Int ((\Int A)'))'$, так как все точки $\Int A$ лежат в $\Int((\Int A)')$\\
    $\Int((\Int A)') \subset (\Int A)'$, так как $(\Int A)'$ -- замкнуто, то все предельные точки точек это множества лежат в $(\Int A)'$, поэтому и $(\Int((\Int A)'))'$ лежит в этом множестве.\\
    Включение в обе стороны есть -- есть равенство.\\
    Получается, что $\Cl(\Int(\Cl(\Int A))) = \Cl(\Int A)$ и $\Int(\Cl(\Int(\Cl A))) = \Int(\Cl A)$, т.е. $\Cl(\Int)$ и $\Int(\Cl)$ больше раза не имеет смысла делать.\\
    Тогда остаются еще $\Cl(\Int(\Cl A))$ и $\Int(\Cl(\Int A))$.\\
    Можно взять множество, в котором есть открытый круг без рациональных точек, закрытый круг, круг без точки и что-нибудь еще похожее, и тогда все наши множества будут чем-то разным на нем.\\
    Тогда у нас получается 7 множеств. $A,\ \Cl A,\ \Int A,\ \Cl(\Int A),\ \Int(\Cl A),\ \Int(\Cl(\Int A)),\ \Cl(\Int(\Cl A))$, а все остальные сводятся к ним.
    \item \textbf{(эта задача необязательна, решивший ее получит 20 баллов)}: придумайте $A,B\subset \mathbb R$ такие, что $A' = B' = \varnothing$, но $(A+B)' = [0,1]$.\\
    Возьмем какой-то $x$. Докажем, что $x = 2^n - 2^m$ имеет конечное число решений, если $n, m \in \mathbb N\cup \{0\}$.\\
    Пусть $x$ делится на $2^k,\ k \in \mathbb N\cup \{0\}$ и $2^k$ -- наибольшая степень 2, которая делит $x$. Тогда наименьшее число из $n$ и $m$ не больше $k$, так как обе части уравнения должны делиться на $2^k$. Тогда и решений конечное число, потому что $m$ и $n$ можно однозначно выразить через друг друга.\\
    Теперь возьмем $A = \{2^n + q_n \mid n \in \mathbb N\}$, где $q_n$ -- какое-то по порядку рациональное число в интервале $(0, 1)$, а $B = \{-2^n\mid n\in \mathbb N\}$.\\
    $A + B = \{2^n + q_n - 2^m\mid n, m\in \mathbb N\}$.\\
    При $m = n$ мы получаем все рациональные числа $(0, 1)$, а их множество предельных точек будет $[0, 1]$. А если $m \neq n$, то мы получем какое-то $x + q_n$, где $x = 2^n - 2^m$. Таких элементов конечное число (мы это доказали ранее), а поэтому ни одна точка из $[x, x+1]$ не будет в $(A+B)'$, кроме $[0, 1]$.\\
    А так как все элементы $A$ и $B$ отличаются друг от друга хотя бы на 1, то $A$ и $B$ дискретны, а потому их множества предельных точек пустые.
\end{enumerate}

\end{document}
















