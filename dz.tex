\documentclass[a4paper]{article}





\usepackage[utf8]{inputenc}
\usepackage[T1]{fontenc}
\usepackage[russian]{babel}
\usepackage{amsmath,amssymb,amsthm}
\usepackage{mathrsfs}
\usepackage[matrix,arrow,curve]{xy}
\usepackage[left=2cm,right=2cm,top=2cm,bottom=2cm,bindingoffset=0cm]{geometry}

\usepackage{multicol}
\setlength{\columnsep}{0.5cm}

\usepackage{enumitem}
%\renewcommand{\labelenumii}{\arabic{enumii}.}

\usepackage{import}


\newtheorem*{rem}{Remark}
\newtheorem{lemma}{Lemma}
\newtheorem{cor}{Corollary}

\def\Im{\mathrm{Im}\,}
\def\eps{\varepsilon}
\def\Int{\mathrm{Int}}
\def\Cl{\mathrm{Cl}}

\def\sh{\mathrm{sh}}
\def\ch{\mathrm{ch}}
\def\th{\mathrm{th}}

\def\arcsh{\mathrm{arcsh}}
\def\arcch{\mathrm{arcch}}
\def\arcth{\mathrm{arcth}}

\def\vphi{\varphi}

\def\pr{\partial}


\begin{document}

\newcommand\HeaderDZ[5]{
\begin{center}
		\textbf{Домашнее задание #3, #2. Дедлайн #4, #5.}\\
		Группа #1.Б10-пу.\\
\end{center}
\vspace{-\baselineskip}
\bigskip
\bigskip
}



\renewcommand{\labelenumii}{\arabic{enumii})}
\renewcommand{\labelenumiii}{\roman{enumiii})}

\HeaderDZ{19}{27.09.19}{3}{08.10.19}{14:00}

\begin{enumerate}
    \item (5) Приведите пример $A, B\subset \mathbb R$ таких, что $A' = B' = \varnothing$, но $(A + B)'\neq \varnothing$, где
    \[
        A+B = \{a+b\ \mid\ a\in A,\ b\in B\}
    \]
    
    \item Точка называется изолированной, если существует окрестность, в которой находится только она сама. Множество называется дискретным, если все его точки изолированные. 
	\begin{enumerate}
		\item (7) Докажите, что дискретное подмножество плоскости всегда не более чем счетно.\\
		Решение: если подмножество дискретно, то у каждой точки из него есть такая окрестность, в которой нет никаких точек этого множества. Тогда возьмем любую рациональную точку из этой окрестности и таким образом сделаем инъекцию в $\mathbb Q^2$, а значит и в $\mathbb N$. 
		\item (10) Постройте пример дискретного множества $A\subset \mathbb R^2$, такого, что $A'$ континуально.
		\item (15) Постройте пример дискретного множества $A\subset\mathbb R$ такого, что $A'$ континуально.
	\end{enumerate}
	
	\item Найдите $A'$, если
    \begin{enumerate}
        \item (8) $A = \{\left(\frac{n}{\sqrt{n^2+m^2}},\frac{m}{\sqrt{n^2+m^2}}\right)\in \mathbb R^2\ \mid\ n,m\in \mathbb N\}$\\
        Решение: $A = \{\left(\frac{1}{\sqrt{1+m^2}},\frac{1}{\sqrt{n^2+1}}\right)\in \mathbb R^2\ \mid\ n,m\in \mathbb N\}$. Если зафиксируем $m$, то $\left(\frac{1}{\sqrt{1+m^2}},\frac{1}{\sqrt{n^2+1}}\right)$ стремится к $\left(\frac{1}{\sqrt{1+m^2}},0\right)$. $A$ является объединением таких последовательностей для всех $m \in \mathbb N$. Так как координаты точки симметричны, то $A'$ -- объединение $\{\left(0,\frac{1}{\sqrt{n^2+1}}\right)\in \mathbb R^2\ \mid\ n \in \mathbb N\}$ и $\{\left(\frac{1}{\sqrt{n^2+1}},0\right)\in \mathbb R^2\ \mid\ n \in \mathbb N\}$. Потому что такие предельные точки у $A$ точно есть, а вне их окрестностей лежит только конечное число элементов.
        \item (8) $A = \{(x,x^n)\in \mathbb R^2\ \mid\ n\in \mathbb N,\ x\in \mathbb R\}$\\
        Решение: $A$ -- объединение графиков функций $f(x) = x^n$. Так как $x \in \mathbb R$. Любая точка такого графика является предельной для $A$. ($x^n$ непрерывна). А вне этих графиков в $A$ нет точек, а значит и предельных тоже нет. Тогда $A' = A$.  
    \end{enumerate}
	
	\item (5) Пусть $A\subset\mathbb R$. Докажите, что существует такая последовательность $a_n\in A$, что $\lim\limits_{n\to +\infty} a_n = \sup A$.\\
	Решение: Если $\sup A$ лежит в $A$, то $a_n = \sup A$. Если не лежит, то $\forall \eps > 0\ \exists\ a\in A:|\sup A - a| < \eps$. Тогда возьмем такую последовательность $a_n = a$, где $|a - \sup A| < \frac{1}{n}$. Тогда с какого-то $n$ все $a_n$ будут попадать в $\eps$-окрестность $\sup A$. 
    \item Предъявите какую-нибудь функцию $N = N(\eps)$, для которой выполняется
    \[
        \forall \eps>0\ \forall n> N(\eps)\ |x_n-\lim\limits_{n\to+\infty} x_n|\leq \eps,
    \]
    если
    \begin{enumerate}
        \item (4) $x_n = \frac{n}{2^{\sqrt{n}}}$\\
        Решение: начиная с некоторого $n:\ n < 2^{\frac{\sqrt{n}}{2}}$. Тогда $\frac{n}{2^{\sqrt{n}}} < \frac{2^{\frac{\sqrt{n}}{2}}}{2^{\sqrt{n}}} = \frac{1}{2^{\frac{\sqrt{n}}{2}}}$ начиная с какого-то $n$. $2^{\frac{-\sqrt{n}}{2}} < \eps \Leftrightarrow \frac{-\sqrt{n}}{2} < \log_2\eps \Leftrightarrow \sqrt{n} > -2\log_2\eps$. $N(\eps) = \left \lceil{4\log^2_2\frac{1}{\eps}}\right \rceil + 1$. А чтобы $N$ было достаточно большим, пускай $N(\eps) = \max(10000,\ \left \lceil{4\log^2_2\frac{1}{\eps}}\right \rceil + 1)$
        \item (4) $x_n = n^{1/n}$\\
        Решение: предел равен 1. (Это докажется, если найду $N(\eps)$).\\ 
        $n^{\frac{1}{n}} > 1$, модуль раскрывается положительно.\\
        $n < (1 + \eps)^n, (1 + \eps)^n > 1 + n\eps + \frac{n(n-1)}{2}\eps^2 > \frac{n}{2}\eps + \frac{n^2}{2}\eps^2$\\
        $\frac{n}{2}\eps + \frac{n^2}{2}\eps^2 > n \Leftrightarrow \frac{n^2}{2}\eps^2 > n(1 - \frac{\eps}{2}) \Leftrightarrow n > \frac{2-\eps}{\eps^2}$\\
        Тогда $N(\eps) = \max(10, \frac{2 - \eps}{\eps^2})$. (10 для больших $\eps$). 
        \item (4) $x_n = \sqrt[3]{n^3 + n^2} - \sqrt[3]{n^3 - n^2}$\\
        Решение: предел равен $\frac{2}{3}$. $|\sqrt[3]{n^3 + n^2} - \sqrt[3]{n^3 - n^2} - \frac{2}{3}| < \eps \Leftrightarrow |\frac{2n^2}{(\sqrt[3]{n^3 + n^2})^2 + \sqrt[3]{n^6 - n^4} + (\sqrt[3]{n^3 - n^2})^2} - \frac{2}{3}| < \eps \Leftrightarrow |\frac{2}{\sqrt[3]{1 + \frac{2}{n} - \frac{1}{n^2}} + \sqrt[3]{1 - \frac{1}{n^2}} + \sqrt[3]{1 - \frac{2}{n} - \frac{1}{n^2}}} - \frac{2}{3}| < \eps \Leftrightarrow 2|\frac{1 - \sqrt[3]{1 + \frac{2}{n} - \frac{1}{n^2}} + 1 - \sqrt[3]{1 - \frac{1}{n^2}} + 1 - \sqrt[3]{1 - \frac{2}{n} - \frac{1}{n^2}}}{3(\sqrt[3]{1 + \frac{2}{n} - \frac{1}{n^2}} + \sqrt[3]{1 - \frac{1}{n^2}} + \sqrt[3]{1 - \frac{2}{n} - \frac{1}{n^2}})}| < \eps$.
        С $n > 10$ точно выполняется, что знаменатель > 1, поэтому:
        $\ 2|\frac{1 - \sqrt[3]{1 + \frac{2}{n} - \frac{1}{n^2}} + 1 - \sqrt[3]{1 - \frac{1}{n^2}} + 1 - \sqrt[3]{1 - \frac{2}{n} - \frac{1}{n^2}}}{3(\sqrt[3]{1 + \frac{2}{n} - \frac{1}{n^2}} + \sqrt[3]{1 - \frac{1}{n^2}} + \sqrt[3]{1 - \frac{2}{n} - \frac{1}{n^2}})}| \le
        2|1 - \sqrt[3]{1 + \frac{2}{n} - \frac{1}{n^2}} + 1 - \sqrt[3]{1 - \frac{1}{n^2}} + 1 - \sqrt[3]{1 - \frac{2}{n} - \frac{1}{n^2}}| \le
        2|3 - \sqrt[3]{1 + \frac{2}{n} - \frac{1}{n^2}}| \le 2|3 - \sqrt[3]{1 + \frac{2}{n} -\frac{1}{n^2}}| \le
        2|3 + \sqrt[3]{\frac{1}{n^2}}|$. $2|3 + \sqrt[3]{\frac{1}{n^2}}| < \eps \Leftrightarrow
        3 + \sqrt[3]{\frac{1}{n^2}} < \frac{\eps}{2} \Leftrightarrow \frac{1}{n^2} < (\frac{\eps}{2} - 3)^3$. При $\eps \ge \frac{3}{2}$ можно брать $n = 10$ и все будет хорошо. Дальше рассматриваем $\eps < \frac{3}{2}$
        \item (4) $x_n = \frac{(n-1)(n-2)\cdot\dots\cdot (n-10)}{(n+1)(n+2)\cdot\dots\cdot(n+10)}$\\
        Предел равен 1.\\
        $|\frac{(n-1)\cdots(n-10) - (n+1)\cdots(n+10)}{(n+1)\cdots(n+10)}| = \frac{(n+1)\cdots(n+10) - (n-1)\cdots(n-10)}{(n+1)\cdots(n+10)} \le \frac{(n+10)^{10} - (n-10)^{10}}{(n+1)^{10}}$ -- точно верно для $n > 10$\\
        $\frac{(n+10)^{10} - (n-10)^{10}}{(n+1)^{10}} = \frac{20((n+10)^9 + \cdots + (n-10)^9)}{(n+1)^{10}} \le \frac{20((n+10)^9 + \cdots + (n-10)^9)}{n^{10}}$\\
        $\frac{20((n+10)^9 + \cdots + (n-10)^9)}{n^{10}} = \frac{20(n+10)^9}{n^{10}} + \cdots + \frac{20(n-10)^9}{n^{10}} \le \frac{200(n+10)^9}{n^{10}}$\\
        $\frac{200(n+10)^9}{n^{10}} \le \frac{200\cdot 252 \cdot 10 \cdot n^9}{n^{10}}$\ Наибольший биноминальный коэф-т равен 252, поэтому это верно.\\
        Тогда $N(\eps) = \max(11, \frac{200\cdot 252 \cdot 10}{\eps})$    
    \end{enumerate}
    
	\item При каких из нижеперечисленных условий числовая последовательность $x_n$ сходится?
	\begin{enumerate}
	    \item (3) $\exists a\in \mathbb R:\ \forall N>0\ \exists \eps>0:\ \forall n\geq\eps\ |x_n - a|<N$\\
        Решение: Если поменять $N$ и $\eps$, то получится определение предела. Сходится.
	    \item (3) $\forall \eps\in \mathbb R\ \exists N:\ \forall n>N\ \eps x_n < 1$\\
	    Решение: Если $\eps > 0$, поделим на $\eps$ и получим $x_n < \frac{1}{\eps}$. При положительных $x_n$ и больших $\eps$ это дает нам стремление к нулю. Если $\eps < 0$, то тоже поделим. $x_n > \frac{1}{\eps}$. И это дает стремление к нулю для отрицательных $x_n$ и большого по модулю $\eps$. Т.е. в итоге $x_n$ стремится к нулю.
	    \item (3) $\forall \eps>0\ \exists N:\ \forall n>N\ \left|\frac{x_1+x_2+\dots + x_n}{n}\right|\leq \eps$\\
	    Решение: $x_n = 1, 0, 1, 0, 0, 1, 0, 0, 0, 1, 0 \dots$ модуль среднего арифметического стремится к нулю, но сама она не сходится.
	    \item (3) $\forall \eps>0\ \exists N:\ \forall n>N\ x_n^2<\frac{1}{1+\eps}$\\
        Решение: $\eps' = \frac{1}{1+\eps}$. $x_n^2 < \eps' \Leftrightarrow |x_n| < \sqrt{\eps'} \Rightarrow x_n$ сходится к 0, так как $\sqrt{\eps'}$ может быть любым положительным числом.
        \item (3) $\forall\eps>0\ \exists N>0:\ \forall n>N\ 0<x_n^n\leq \frac{\eps}{n}$\\
	    Решение: $\forall\eps>0\ \exists N>0:\ \forall n>N\ 0<x_n^n\leq \frac{\eps}{n} \Rightarrow 0<x_n \leq \sqrt[n]{\frac{\eps}{n}}$.\\
        Заметим: $\frac{1}{\sqrt[n]{n}} = (\sqrt[n]{n})^{-1}, \sqrt[n]{n} \rightarrow 1 \Rightarrow \frac{1}{\sqrt[n]{n}} \rightarrow 1$\\
        Докажем, что полученная последовательность стремится к 1.\\
        $\forall \eps > 0\ {\sqrt[n]{\frac{\eps}{n}}} = \sqrt[n]{\eps}\cdot \sqrt[n]{\frac{1}{n}}$. Для $\eps < 1$ верно: $\sqrt[n]{\eps}\cdot \sqrt[n]{\frac{1}{n}} \rightarrow 1\cdot 1 = 1$. При этом $\sqrt[n]{\frac{\eps}{n}} < 1$.\\
        Пусть у нас $\eps$ фиксированный и $< 1$. Тогда у нас $|1 - x_n| < |1 - \sqrt[n]{\frac{\eps}{n}}| = 1 - \sqrt[n]{\frac{\eps}{n}} \rightarrow 0$. Тогда для $\forall \eps' > 0\ \exists N:\ n > N\ |1 - x_n| < \eps'$. Тогда $x_n \rightarrow 1$.
        \item (3) $\exists N:\ \forall n>0\ \frac{1}{n}\leq (x_n - 1)^n \leq n$\\
        Решение: $\frac{1}{n}\leq (x_n - 1)^n \leq n \Leftrightarrow \frac{1}{\sqrt[n]{n}} \le x_n - 1 \le \sqrt[n]{n}$. Левая и правая части стремятся к 1. По т. о двух миллиционерах предел есть и равен 2.
	\end{enumerate}
    
    \item Докажите, что последовательность $x_n$ сходится, и вычислите ее предел, если
    \begin{enumerate}
        \item (7) $x_1=1,\quad x_{n+1}=\frac{x_n + 1}{x_n + 2}$\\
        Решение:\\ 
        Заметим, что $x_n > 0$. Найдем, когда убывает\\
        $x_n > \frac{x_n + 1}{x_n + 2} \Leftrightarrow x_n^2 + 2x_n > x_n + 1 \Leftrightarrow x_n^2 + x_n - 1 > 0$\\
        Заметим, что пока $x_n > \frac{\sqrt{5} - 1}{2}$, $x_n$ убывает.\\
        Найдем, когда $x_{n+1} \le \frac{\sqrt{5} - 1}{2}$\\
        $\frac{x_n + 1}{x_n + 2} \le \frac{\sqrt{5} - 1}{2} \Leftrightarrow 2x_n + 2 \le \sqrt{5}x_n - x_n + 2\sqrt{5} - 2 \Leftrightarrow x_n(3 - \sqrt{5}) \le 2\sqrt{5} - 4 \Leftrightarrow x_n \le \frac{2\sqrt{5} - 4}{3 - \sqrt{5}}$\\
        Сравним теперь это с $\frac{\sqrt{5} - 1}{2}$\\
        $\frac{\sqrt{5} - 1}{2} - \frac{2\sqrt{5} - 4}{3 - \sqrt{5}} = \frac{3\sqrt{5} - 3 - 5 + \sqrt{5} - 4\sqrt{5} + 8}{2(3 - \sqrt{5})} = 0$.\\
        Получается, что $x_{n+1} \le \frac{\sqrt{5} - 1}{2}$ только, если $x_{n} \le \frac{\sqrt{5} - 1}{2}$.\\
        $1 > \frac{\sqrt{5} - 1}{2}, x_n$ ограничена снизу. Значит, она сходится. Тогда пусть $x_n \rightarrow a$.\\
        $a = \frac{a+1}{a+2} \Rightarrow a = \frac{\sqrt{5} - 1}{2}$.\\
        Предел равен $\frac{\sqrt{5} - 1}{2}$
        \item (10) $x_1 = 2,\ x_2 = 1,\quad x_{n+2} = \sqrt[3]{x_{n+1}^2x_n}$\\
        Решение:\\
        $\frac{x_{n+3}}{x_{n+2}} = \frac{\sqrt[3]{x_{n+2}^2x_{n+1}}}{x_{n+2}} =
        \sqrt[3]{\frac{x_{n+1}}{x_{n+2}}} \Rightarrow \frac{x_{n+3}}{x_{n+2}} =
        \sqrt[9]{\frac{x_{n+1}}{x_n}} \Rightarrow \frac{x_{n+4}}{x_{n+3}} = 
        \sqrt[9]{\frac{x_{n+2}}{x_{n+1}}} \Rightarrow \frac{x_{n+4}}{x_{n+2}} = 
        \sqrt[9]{\frac{x_{n+2}}{x_{n}}}$\\
        Аналогично: $\frac{x_{n+5}}{x_{n+3}} = \sqrt[9]{\frac{x_{n+3}}{x_{n+1}}}$\\
        Тогда: $x_5=x_3\cdot \frac{x_5}{x_3} = x_3 \sqrt[9]{\frac{x_3}{x_1}}$, $x_7=x_5\cdot \frac{x_7}{x_5} = x_5 \sqrt[9]{\frac{x_5}{x_3}} = x_5 \sqrt[81]{\frac{x_3}{x_1}} = x_3 \sqrt[9]{\frac{x_3}{x_1}}\sqrt[81]{\frac{x_3}{x_1}}$ и тд.\\
        По аналогии: $x_{2n+1} = x_3(\frac{x_3}{x_1})^{\sum_{k=1}^{n-1} \frac{1}{9^k}}$. $n_{+\infty} = x_3(\frac{x_3}{x_1})^\frac{1}{8}$ и также $x_{2n} = x_4(\frac{x_4}{x_2})^{\sum_{k=1}^{n-2}\frac{1}{9^k}} \Rightarrow x_{+\infty} = x_4(\frac{x_4}{x_2})^{\frac{1}{8}}$\\
        $x_3 = \sqrt[3]{2}, \frac{x_3}{x_1} = 2^{-\frac{2}{3}}, x_4 = \sqrt[3]{4^{\frac{1}{3}}} = 2^{\frac{2}{9}}, \frac{x_4}{x_2} = 2^{\frac{2}{9}}.$\\
        $2^{\frac{1}{3}}\cdot 2^{-\frac{2\cdot 1}{3\cdot 8}} = 2^{\frac{1}{3}}\cdot 2^{-\frac{1}{12}} = 2^{\frac{1}{4}} = 2^{\frac{2\cdot 9}{9\cdot 8}} = 2^{\frac{2}{9}(1 + \frac{1}{8})} = 2^\frac{2}{9}\cdot 2^{\frac{2\cdot 1}{9\cdot 8}} \Rightarrow$ пределы последовательностей из четных и нечетных $x_n$ совпадают $\Rightarrow x_n$ сходится и имеет предел $2^\frac{1}{4}$. 
    \end{enumerate}
    
    \item (10) При каких $x_1\in\mathbb{R}$ последовательность, заданная рекуррентно $x_{n+1}=x_n(2-x_n)$, $n\geqslant1$, сходится? Если сходится, найдите предел.\\
    Решение:\\
    Изучим монотонность:\\
    $x_n(2 - x_n) \wedge x_n \Leftrightarrow 0\ \wedge x_n^2 - x_n \Leftrightarrow 0\ \wedge x_n(x_n - 1) \Rightarrow$ При $x \in [0, 1] x_n$ возрастает, иначе убывает.\\
    Рассмотрим $x_1 < 0$. Тогда вся последовательность $< 0$. $\frac{x_{n+1}}{x_n} = 2 - x_n$. Отношение растет, $> 1$ $\Rightarrow x_n$ уходит в $-\infty$, предела нет.\\
    Если $x_1 = 0$, то $x_2 = 0$ и тд. Предел есть, равен 0.\\
    Если $x_1 = 1$, то $x_2 = 1$ и тд. Предел есть, равен 1.\\
    Найдем, когда $x_{n+1} > 1$\\
    $x_n(2-x_n) > 1 \Leftrightarrow 0 > x_n^2 - 2x_n + 1 \Leftrightarrow 0 > (x_n - 1)^2 \Rightarrow$. $x_{n+1}$ всегда ограничен сверху 1.\\
    Найдем, когда $x_{n+1} > 0$\\
    $x_n(2-x_n) > 0$ Получается, что $x_{n+1} > 0 \Leftrightarrow x_n \in (0, 2)$\\
    Тогда, если $x_1 > 2$, то предела опять нет. Если равен 2, то предел 0.\\
    Если $x_1 \in (0, 2)\!\setminus\{1\}$, то $x_2 \in (0, 1)$. И все остальные $x_n$ тоже. Тогда у нас $x_n$ ограничена и возрастает, значит предел есть.\\
    Тогда пусть предел равен $a$. Подставим $a$. $a = 2a - a^2 \Rightarrow a = 1$, тк нулем быть не может.

\end{enumerate}

\end{document}
















