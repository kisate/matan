\documentclass[a4paper]{article}





\usepackage[utf8]{inputenc}
\usepackage[T1]{fontenc}
\usepackage[russian]{babel}
\usepackage{amsmath,amssymb,amsthm}
\usepackage{mathrsfs}
\usepackage[matrix,arrow,curve]{xy}
\usepackage[left=2cm,right=2cm,top=2cm,bottom=2cm,bindingoffset=0cm]{geometry}

\usepackage{multicol}
\setlength{\columnsep}{0.5cm}

\usepackage{enumitem}
%\renewcommand{\labelenumii}{\arabic{enumii}.}

\usepackage{import}


\newtheorem*{rem}{Remark}
\newtheorem{lemma}{Lemma}
\newtheorem{cor}{Corollary}

\def\Im{\mathrm{Im}\,}
\def\eps{\varepsilon}
\def\Int{\mathrm{Int}}
\def\Cl{\mathrm{Cl}}

\def\sh{\mathrm{sh}}
\def\ch{\mathrm{ch}}
\def\th{\mathrm{th}}

\def\arcsh{\mathrm{arcsh}}
\def\arcch{\mathrm{arcch}}
\def\arcth{\mathrm{arcth}}

\def\vphi{\varphi}

\def\pr{\partial}


\begin{document}

\newcommand\HeaderDZ[1]{
\begin{center}
		\textbf{Подготовка к КР#1.}
\end{center}
\vspace{-\baselineskip}
\bigskip
\bigskip
}



\renewcommand{\labelenumii}{\arabic{enumii})}
\renewcommand{\labelenumiii}{\roman{enumiii})}

\HeaderDZ{1}

\begin{enumerate}
    \item $A = \{x \in [0, 1]\mid\ x = 0,a_1a_2a_3\cdots, a_i\cdot i = 0 \mod 10\}$\\
    Элементы $A$ -- это вещественные числа от 0 до 1, у которых каждая четная цифра либо 0, либо 5, каждая пятая должна быть четной, а каждая десятая -- любой. Все остальные -- 0.\\
    $\inf A = 0$, так как 0 получается просто, когда все $a_i$ равны 0, а чисел меньше нуля в $A$ нет.\\
    $\sup A = 0,(0505805059)$. Большего числа в $A$ нет, значит это супремум.\\
    Рассмотрим $\eps > 0$. $\exists n\in \mathbb {N} :\ \eps' = 10^{-n} < \eps$\\
    Тогда $\eps'$-окрестность любого числа лежит в $\eps$-окрестности этого числа. И тогда все цифры, начиная с $n+1$, можно менять как угодно, а число все равно останется в $\eps$-окрестности.\\
    $\Int A = \varnothing$, так как по соображению выше в любой окрестности $a \in A$ будет число, у которого на каком-то четном месте стоит не 0, а 1, и такое число не будет лежать в $A$.\\
    Так как у нас каждая десятая цифра может быть любой, то $A \subset A'$. Потому что все по тому же соображению мы можем поменять какую-то цифру на месте, кратном десяти и получить число из $A$, которое лежит в любой окрестности $a \in A$.\\
    Теперь возьмем какое-то число не из $A$. Тогда у него есть хотя бы одна неправильная цифра. Пускай она под номером $n$. Тогда в любой $\eps$-окрестности этого числа, если $\eps < 10^{-(n+2)}$, все числа будут отличаться не больше, чем на $10^{-(n+2)}$, т.е. они будут отличаться только в цифрах, номер которых больше или равен $n+1$. Тогда у них всегда будет неправильная цифра на $n$-ом месте, а значит они все не будут лежать в $A$. Тогда $A' = A$.\\
    Тогда и $\Cl A = A$ и $\partial A = A$.

    \item $x_1 = a$, $x_{n+1} = x_n^2 + \frac{x_n}{2}$.\\
    Рассмотрим, когда убывает/возрастает.\\
    $x_n < x_n^2 + \frac{x_n}{2} \Leftrightarrow 2x_n^2 - x_n > 0 \Leftrightarrow x_n(x_n - \frac{1}{2}) > 0$.\\
    $x_n^2 + \frac{x_n}{2} > 0 \Leftrightarrow x_n(x_n + \frac{1}{2}) > 0$.\\
    $x_n^2 + \frac{x_n}{2} > \frac{1}{2} \Leftrightarrow 2x_n^2 + x_n - 1 > 0 \Leftrightarrow (x_n + 1)(x_n - \frac{1}{2}) > 0$.\\
    Если $x_n > \frac{1}{2}$, то $x_{n+1} > x_n$. $\frac{x_n^2 + \frac{x_n}{2}}{x_n} = x_n + \frac{1}{2}$, т.е. $\{x_n\} \rightarrow +\infty$. \\
    Т.е. при $a > \frac{1}{2}$ предел равен $+\infty$.\\
    Если $x_n \in (0, \frac{1}{2})$, то $x_{n+1} < x_n$, но при этом $x_{n+1} > 0$. Тогда ${x_n}$ монотонна и ограничена $\Rightarrow$ есть предел. Пускай он равен $b$. $b = b^2 + \frac{b}{2}$, подходит только $b=0$. \\
    Получется, что при $a \in (0, \frac{1}{2})$ предел равен 0.\\
    Если $x_n \in (-\frac{1}{2}, 0)$, то $x_{n+1} > x_n$, но $x_{n+1} < 0$. Значит последовательность монотонна и ограничена, поэтмому есть предел. Из $b = b^2 + \frac{b}{2}$ следует, что предел равен 0.\\ Т.е. при $a \in (-\frac{1}{2}, 0)$ $\{x_n\} \rightarrow 0$.\\
    Если $x_n \in (-1, -\frac{1}{2})$, то $x_{n+1} > x_n$. $x_{n+1} > 0$ и $x_{n+1} < \frac{1}{2}$. Т.е. попадаем во второй случай.\\
    Тогда при $a \in (-1, -\frac{1}{2})$ $\{x_n\}\rightarrow 0$.\\
    Если $x_{n} < -1$, то $x_{n+1} > x_n$. $x_{n+1} > 0$ и $x_{n+1} > \frac{1}{2}$, т.е опять предел $+\infty$. \\
    Тогда при $a < -1$ $\{x_n\}\rightarrow+\infty$.\\
    $a = 0$, тогда $x_n = 0$, $\{x_n\}\rightarrow 0$.\\
    $a = \frac{1}{2}$, тогда $x_n = \frac{1}{2}$. $\{x_n\}\rightarrow\frac{1}{2}$.\\
    $a = -\frac{1}{2}$, $x_n = 0$. $\{x_n\}\rightarrow 0$.\\
    $a = -1$, $x_2 = \frac{1}{2} \Rightarrow \{x_n\}\rightarrow\frac{1}{2}$.\\
    Теперь $N(\eps)$.\\ 
    $a = \frac{1}{2} + \alpha$, $\alpha > 0$:\\
    $\frac{x_n+1}{x_n} = x_{n+1} + \frac{1}{2} \ge 1 + \alpha$.\\
    $x_{n+1} = a\cdot (1+\alpha)^n \ge a\cdot(1 + n\alpha)$\\
    Надо чтобы $x_{n} \ge \eps$, $\eps > 0$.\\
    $\eps \le a\cdot(1 + n\alpha) \Leftrightarrow \eps - a \le an\alpha \Leftrightarrow n \ge \frac{\eps - a}{a\alpha}$.\\
    Тогда $N(\eps) = \max(2, \frac{\eps - a}{a\alpha} + 1)$, если $a > \frac{1}{2}$.\\
    При $a \in \{\frac{1}{2}, 0, -\frac{1}{2}, -1\}\ N(\eps) = 1$.\\
    $a = -1 - \alpha,\ \alpha > 0:$\\
    $x_{2} = a^2 + \frac{a}{2} = (1+\alpha)^2 - \frac{1+\alpha}{2} = 1 + 2\alpha + \alpha^2 - \frac{1}{2} - \frac{\alpha}{2} = \frac{1}{2} + \alpha^2 + \frac{3}{2}\alpha > \frac{1}{2} + \alpha$. Тогда $\frac{x_{n+1}}{x_n} = x_n + \frac{1}{2} > 1 + \alpha$. Тогда $x_{n+2} > x_2\cdot(1 + \alpha)^{n} \ge x_2(1 + n\alpha)$.\\
    Хотим, чтобы $x_{n+2} > \eps > 0.$\\
    $x_2(1+n\alpha) > \eps \Leftrightarrow n > \frac{\eps - x_2}{x_2n}$, где $x_2 = a^2 + \frac{a}{2}$.\\
    Тогда $N(\eps) = \max(3, \frac{\eps - x_2}{x_2n} + 2)$\\
    $a \in (0, \frac{1}{2}), a = \frac{1}{2} - \alpha, \alpha \in (0, \frac{1}{2})$:
    $\frac{x_{n+1}}{x_n} = x_n + \frac{1}{2} \le 1 - \alpha$\\
    $x_{n+1} \le a\cdot(1 - \alpha)^n$.\\
    Так как $x_n > 0$, то нужно, чтобы $x_n < \eps$.\\
    $a\cdot(1-\alpha)^n < \eps \Leftrightarrow n\log_a(a(1-a)) > \log_a\eps \Leftrightarrow n > \frac{\log_a\eps}{\log_a(a(1-a))}$\\
    $N(\eps) = \max(2, \frac{\log_a\eps}{\log_a(a(1-a))})$.\\
    Я хочу спать, поэтому скажу, что для остальных $a$ аналогично.

    \item 
    \begin{enumerate}
        \item $\lim\limits_{n\rightarrow\infty}(\sqrt{\frac{1}{n}} + \cdots + \sqrt{\frac{n}{n}} - \frac{2n}{3}) = \lim\limits_{n\rightarrow\infty}(\frac{1 + \sqrt{2} + \cdots + \sqrt{n} - \frac{2n\sqrt{n}}{3}}{\sqrt{n}})$. $\sqrt{n}\rightarrow+\infty$.\\
        $\lim\limits_{n\rightarrow\infty}(\frac{1 + \sqrt{2} + \cdots + \sqrt{n} - \frac{2n\sqrt{n}}{3}}{\sqrt{n}}) = \lim\limits_{n\rightarrow\infty}\frac{\sqrt{n+1} - \frac{2(n+1)\sqrt{n+1}}{3} + \frac{2n\sqrt{n}}{3}}{\sqrt{n+1} - \sqrt{n}} = \lim\limits_{n\rightarrow\infty}\frac{\frac{1}{3}\sqrt{n+1}-\frac{2n}{3}(\sqrt{n+1} - \sqrt{n})}{\sqrt{n+1}-\sqrt{n}}=$\\
        $= \lim\limits_{n\rightarrow\infty}(\frac{\frac{1}{3}\sqrt{n+1}}{\sqrt{n+1}-\sqrt{n}} - \frac{2n}{3}) = \lim\limits_{n\rightarrow\infty}(\frac{\frac{1}{3}\sqrt{n+1}(\sqrt{n+1} + \sqrt{n})}{1} - \frac{2n}{3}) = \lim\limits_{n\rightarrow\infty}(\frac{1}{3}(n+1) + \frac{1}{3}\sqrt{n^2 + n} - \frac{2n}{3}) = $\\
        $=\lim\limits_{n\rightarrow\infty}(\frac{1}{3}\sqrt{n^2 + n} - \frac{n}{3} + \frac{1}{3}) = \frac{1}{3}\lim\limits_{n\rightarrow\infty}(\frac{n^2 + n - n^2}{\sqrt{n^2 + n} + n} + 1) = \frac{1}{3}\cdot(\frac{1}{2} + 1) = \frac{1}{2}$ 
        \item $\lim\limits_{n\rightarrow\infty}\frac{1}{n}\sum_{k=0}^{n-1}(1 + \frac{1}{n})^k = \lim\limits_{n\rightarrow\infty}\frac{1}{n}\sum_{k=0}^{n-1}(1+\frac{1}{n})^k = \lim\limits_{n\rightarrow\infty}(\frac{1}{n}\cdot \frac{(1 + \frac{1}{n})^n - 1}{\frac{1}{n}}) =\lim\limits_{n\rightarrow\infty}(1 + \frac{1}{n})^n - 1 =e - 1$
    \end{enumerate}
\end{enumerate}

\end{document}
















