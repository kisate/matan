\documentclass[a4paper]{article}




\usepackage{forloop}

\usepackage[utf8]{inputenc}
\usepackage[T1]{fontenc}
\usepackage[russian]{babel}
\usepackage{amsmath,amssymb,amsthm}
\usepackage{mathrsfs}
\usepackage[matrix,arrow,curve]{xy}
\usepackage[left=2cm,right=2cm,top=2cm,bottom=2cm,bindingoffset=0cm]{geometry}

\usepackage{multicol}
\setlength{\columnsep}{0.5cm}

\usepackage{enumitem}
%\renewcommand{\labelenumii}{\arabic{enumii}.}

\usepackage{import}
\usepackage{comment}


\newtheorem*{rem}{Remark}
\newtheorem{lemma}{Lemma}
\usepackage{forloop}
\newtheorem{cor}{Corollary}

\def\Im{\mathrm{Im}\,}
\def\eps{\varepsilon}
\def\Int{\mathrm{Int}}
\def\Cl{\mathrm{Cl}}

\def\sh{\mathrm{sh}}
\def\ch{\mathrm{ch}}
\def\th{\mathrm{th}}

\def\arcsh{\mathrm{arcsh}}
\def\arcch{\mathrm{arcch}}
\def\arcth{\mathrm{arcth}}

\def\vphi{\varphi}

\def\pr{\partial}
\def\Int{\mathrm{Int}}
\def\Cl{\mathrm{Cl}}

%beta-function
\def\B{\mathrm{B}}

\begin{document}




\renewcommand{\labelenumii}{\arabic{enumii})}
\renewcommand{\labelenumiii}{\roman{enumiii})}




\begin{enumerate}
    
    \item В этом задании мы докажем, что 
    \[
        \sum\limits_{k = 1}^n \frac{1}{k} = \log n + \gamma + \frac{1}{2n} + o\left( \frac{1}{n} \right).
    \]
    Пререквизиты, которые нам понадобятся:
    \begin{enumerate}[label=(\arabic*)]
        \item Теорема Штольца,
        \item наревенство $x-\frac{x^2}{2}\leq \log (1+x)\leq x$ (будет доказано на практике),
        \item формула Тейлора для $\log(1+x)$ второго порядка, чтобы заменять $\log\left(1 + \frac{1}{n}\right)$ на более удобное выражение.
    \end{enumerate}
    Заведем обозначение $H_n = \sum\limits_{k = 1}^n \frac{1}{k}$ для удобства.
    \begin{enumerate}
        \item (7) Найдите предел $\lim\limits_{n\to +\infty}\frac{H_n}{\log n}$
        \item (10) Покажите, что последовательность $\{H_n - \log n\}$ сходится.
        
        \noindent\emph{Подсказка:} покажите, что последовательность $H_{n-1} - \log n$ монотонно возрастает, а последовательность $H_n - \log n$ монотонно убывает.
        
        \item (10) Пусть $\gamma:= \lim\limits_{n\to +\infty} (H_n - \log n)$. С помощью теоремы Штольца покажите, что
        \[
            \lim\limits_{n\to +\infty}n(H_n - \log n - \gamma) = \frac{1}{2}.
        \]
    \end{enumerate}
    
    \item Выпишите формулу Тейлора для функции $f$ в $0$ до порядка $n$, если
    \begin{enumerate}
        \item (5) $f(x) = \frac{1}{(x+1)(x-2)}$
        \item (5) $f(x) = (x-1)e^{x/2}$
        \item (5) $f(x) = \begin{cases}e^{-1/x^2},\quad x\neq 0,\\ 0,\quad x=0\end{cases}$
    \end{enumerate}
    
    \item Найдите следующие пределы:
    \begin{enumerate}
        \item (5) $\lim\limits_{x\to 0}\frac{\log(x+\cos(x)) - x}{x^2}$\\
        $\lim\limits_{x\to 0}\frac{\log(x+\cos(x)) - x}{x^2} = \lim\limits_{x\to 0}\frac{\log(x + 1 - \frac{x^2}{2} + \mathcal O(x^4)) - x}{x^2} = 
        \lim\limits_{x\to 0}\frac{x - \frac{x^2}{2} + \mathcal O(x^4) + \frac{(x - \frac{x^2}{2} + \mathcal O(x^4))^2}{2} + \mathcal O((x - \frac{x^2}{2} + \mathcal O(x^4))^3) - x}{x^2} =\\= \lim\limits_{x\to 0}\frac{- \frac{x^2}{2} - \frac{x^2}{2} + \mathcal O(x^3)}{x^2} = -1$
        \item (5) $\lim\limits_{x\to 0}\frac{\sqrt{1+x} + \sqrt[3]{1+x} - 2\sqrt[4]{1-x}}{x}$\\
        $\lim\limits_{x\to 0}\frac{\sqrt{1+x} + \sqrt[3]{1+x} - 2\sqrt[4]{1-x}}{x} = \lim\limits_{x\to 0}\frac{1 + \frac{x}{2} + \mathcal O(x^2) + 1 + \frac{x}{3} + \mathcal O(x^2) - 2 + \frac{x}{2} + \mathcal O(x^2)}{x} = \frac{4}{3}$ 
        \item (5) $\lim\limits_{x\to 0}\frac{\sqrt{1+x} - \sqrt[3]{1+x} - \sqrt[6]{1+x}}{x^2}$
        \item (5) $\lim_{x\to 0}\frac{e^x - \sqrt{1 + 2x + 2x^2}}{x + \tg(x) - \sin (2x)}$
        \item (5) $\lim\limits_{x\to 0}(\cos(x))^{\frac{1}{x^2}}$\\
        $\lim\limits_{x\to 0}(\cos(x))^{\frac{1}{x^2}} = \lim\limits_{x\to 0}(e^{\frac{1}{x^2}\ln cosx}) = \lim\limits_{x\to 0}(e^{\frac{1}{x^2}\ln (1 - \frac{x^2}{2} + \mathcal O(x^4))}) = \lim\limits_{x\to 0}(e^{\frac{1}{x^2}(-\frac{x^2}{2} + \mathcal O(x^4))}) = e^{-\frac{1}{2}}$
        \item (5) $\lim\limits_{x\to 1}\frac{\sin(\pi x)}{\log x}$\\
        $x = y + 1$\\
        $\lim\limits_{x\to 1}\frac{\sin(\pi x)}{\log x} = \lim\limits_{y\to 0}\frac{\sin(\pi y + \pi)}{\log (y + 1)} = \lim\limits_{y\to 0} \frac{0 - \pi y + \mathcal O(y^3)}{y} = -\pi$
        \item (5) $\lim\limits_{x\to 0}\frac{(1 + x)^{\frac{1}{x}} - e}{x}$\\
        $\lim\limits_{x\to 0}\frac{(1 + x)^{\frac{1}{x}} - e}{x}$ = $\lim\limits_{x\to 0}\frac{e^{\frac{1}{x}\ln(1 + x)} - e}{x}$ = $\lim\limits_{x\to 0}\frac{e^{\frac{1}{x}(x - \frac{x^2}{2} + \mathcal O(x^3))} - e}{x}$ = $\lim\limits_{x\to 0}\frac{e\cdot e^\frac{-x}{2}\cdot e^{\mathcal O(x^2)} - e}{x}$ = $e\lim\limits_{x\to 0}\frac{e^\frac{-x}{2}\cdot e^{\mathcal O(x^2)} - 1}{x}$ =\\= $e\lim\limits_{x\to 0}\frac{(1 - \frac{x}{2} + \mathcal O(x^2))\cdot (1 + \mathcal O(x^2)) - 1}{x}$ = $e\lim\limits_{x\to 0}\frac{1 - \frac{x}{2} + \mathcal O(x^2) + \mathcal O(x^2) + \mathcal O(x^3) + \mathcal O(x^4) - 1}{x}$ = $e\lim\limits_{x\to 0}\frac{-\frac{x}{2} + \mathcal O(x^2)}{x}$ = $-\frac{e}{2}$
        \item (5) $\lim\limits_{x\to 0}\frac{\ln (x+\sqrt{x^2+1})-x}{(e^x-e^{\sin x})}$\\
        $\lim\limits_{x\to 0}\frac{\ln (x+\sqrt{x^2+1})-x}{(e^x-e^{\sin x})}$ = 
        $\lim\limits_{x\to 0}\frac{\ln (x + 1 + \frac{x^2}{2} + \mathcal O(x^4))-x}{(1 + x + \frac{x^2}{2} + \frac{x^3}{6} + \mathcal O(x^4) - (1 + x + \frac{x^2}{2} + \mathcal O(x^4)))}$ =\\=
         $\lim\limits_{x\to 0}\frac{x + \frac{x^2}{2} + \mathcal O(x^4)  - \frac{(x + \frac{x^2}{2} + \mathcal O(x^4))^2}{2} + \frac{(x + \mathcal O(x^2))^3}{3} + \mathcal O(x^4)-x}{\frac{x^3}{6} + \mathcal O(x^4)}$ =\\=
         $\lim\limits_{x\to 0}\frac{\frac{x^2}{2} + \mathcal O(x^4)  - \frac{x^2 + x^3 + \mathcal O(x^4)}{2} + \frac{x^3 + \mathcal O(x^4)}{3} + \mathcal O(x^4)}{\frac{x^3}{6} + \mathcal O(x^4)}$ = 
         $\lim\limits_{x\to 0}\frac{-\frac{x^3}{2} + \frac{x^3}{3} + \mathcal O(x^4)}{\frac{x^3}{6} + \mathcal O(x^4)}$ = 
         $\lim\limits_{x\to 0}\frac{-\frac{1}{2} + \frac{1}{3} + \mathcal O(x)}{\frac{1}{6} + \mathcal O(x)}$ = 
         $-3 + 2$ = $-1$
        \item (7) $\lim\limits_{x\to 0}\frac{x\cdot e^{(e^x-1)}+1-\frac{1}{1-x}}{\sin^2(x)-x^2\cos^2(x)}$
        \item (7) $\lim\limits_{x\to 0} \frac{e^x+\sin x - \cos x + 2\ln(1-x)}{x(\sqrt{1+2x}-\sqrt[3]{1+3x})}$\\
        $\lim\limits_{x\to 0} \frac{e^x+\sin x - \cos x + 2\ln(1-x)}{x(\sqrt{1+2x}-\sqrt[3]{1+3x})}$ = $\lim\limits_{x\to 0} \frac{1 + x + \frac{x^2}{2} + \frac{x^3}{6} + \mathcal O(x^4) + x - \frac{x^3}{6} + \mathcal O(x^5) - 1 + \frac{x^2}{2} + \mathcal O(x^4) + 2(-x - \frac{x^2}{2} - \frac{x^3}{3} + \mathcal O(x^4))}{x(1 + x - \frac{(2x)^2}{8} + \mathcal O(x^3) - (1 + x - x^2 + \mathcal O(x^3))}$ =\\=
         $\lim\limits_{x\to 0} \frac{2(-\frac{x^3}{3}) + \mathcal O(x^4) }{x(\frac{x^2}{2} + \mathcal O(x^3))}$ = 
         $\lim\limits_{x\to 0} \frac{-\frac{2}{3} + \mathcal O(x^4) }{\frac{1}{2} + \mathcal O(x)}$ = $-\frac{4}{3}$
    \end{enumerate}
    
    \item (7) Пусть $f:[0,1]\to \mathbb R$ --- непрерывная функция и $\max_{0\leq x\leq 1}\cos (f(x))=\frac{1}{2}$. Покажите, что для всякого $a\in \mathbb R$ уравнения $f(x) = a$ и $f(x) = a + 5$ не могут одновременно иметь решение.\\
    Если $f(x) = a$ и $f(x) = a + 5$, то, тк $f(x)$ непрерывна, то она принимает и все значения между этими точками. По условию задачи $\cos(f(x)) \le \frac{1}{2}$, т.е. $f(x) \in [\frac{\pi}{3} + 2\pi k, \frac{5\pi}{3} + 2\pi k]$. 
    Получается, что все числа между $a$ и $a + 5$ должны лежать между $\frac{\pi}{3}$ и $\frac{5\pi}{3}$ на тригонометрической окружности, но $\frac{4\pi}{3} < 5$, а значит часть этих значений будет вне нужной области, что противоречит условию.

\item (7) Пусть $f:[0,1]\to \mathbb R$ --- непрерывная функция и $\max\limits_{0\leq x\leq 1}f(x)^3 = \max\limits_{0\leq x\leq 1}\exp(-f(x) + 1)$. Покажите, что уравнение $f(x) = 1$ имеет решение.\\
$(f(x)^3)' = 3f(x)^2f(x)'$, $(\exp(-f(x) + 1))' = -e^{1-f(x)}f(x)'$

\item (7) Пусть  $f:[0,1]\to \mathbb R$ --- непрерывная функция, $f(0)\neq f(1)$ и $f$ принимает каждое свое значение не больше трех раз. Докажите, что некоторое значение принимается функцией $f$ нечетное количество раз.
    
    
    
\end{enumerate}

\end{document}
