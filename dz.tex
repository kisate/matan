\documentclass[a4paper]{article}





\usepackage[utf8]{inputenc}
\usepackage[T1]{fontenc}
\usepackage[russian]{babel}
\usepackage{amsmath,amssymb,amsthm}
\usepackage{mathrsfs}
\usepackage[matrix,arrow,curve]{xy}
\usepackage[left=2cm,right=2cm,top=2cm,bottom=2cm,bindingoffset=0cm]{geometry}

\usepackage{multicol}
\setlength{\columnsep}{0.5cm}

\usepackage{enumitem}
%\renewcommand{\labelenumii}{\arabic{enumii}.}

\usepackage{import}


\newtheorem*{rem}{Remark}
\newtheorem{lemma}{Lemma}
\newtheorem{cor}{Corollary}

\def\Im{\mathrm{Im}\,}
\def\eps{\varepsilon}
\def\Int{\mathrm{Int}}
\def\Cl{\mathrm{Cl}}

\def\sh{\mathrm{sh}}
\def\ch{\mathrm{ch}}
\def\th{\mathrm{th}}

\def\arcsh{\mathrm{arcsh}}
\def\arcch{\mathrm{arcch}}
\def\arcth{\mathrm{arcth}}

\def\vphi{\varphi}

\def\pr{\partial}


\begin{document}

\newcommand\HeaderDZ[5]{
\begin{center}
		\textbf{Домашнее задание #3, #2. Дедлайн #4, #5.}\\
		Группа #1.Б10-пу.\\
\end{center}
\vspace{-\baselineskip}
\bigskip
\bigskip
}



\renewcommand{\labelenumii}{\arabic{enumii})}
\renewcommand{\labelenumiii}{\roman{enumiii})}

\HeaderDZ{19}{27.09.19}{3}{08.10.19}{14:00}

\begin{enumerate}
    \item (5) Приведите пример $A, B\subset \mathbb R$ таких, что $A' = B' = \varnothing$, но $(A + B)'\neq \varnothing$, где
    \[
        A+B = \{a+b\ \mid\ a\in A,\ b\in B\}
    \]
    
    \item Точка называется изолированной, если существует окрестность, в которой находится только она сама. Множество называется дискретным, если все его точки изолированные. 
	\begin{enumerate}
		\item (7) Докажите, что дискретное подмножество плоскости всегда не более чем счетно.\\
		Решение: если подмножество дискретно, то у каждой точки из него есть такая окрестность, в которой нет никаких точек этого множества. Тогда возьмем любую рациональную точку из этой окрестности и таким образом сделаем инъекцию в $\mathbb Q^2$, а значит и в $\mathbb N$. 
		\item (10) Постройте пример дискретного множества $A\subset \mathbb R^2$, такого, что $A'$ континуально.
		\item (15) Постройте пример дискретного множества $A\subset\mathbb R$ такого, что $A'$ континуально.
	\end{enumerate}
	
	\item Найдите $A'$, если
    \begin{enumerate}
        \item (8) $A = \{\left(\frac{n}{\sqrt{n^2+m^2}},\frac{m}{\sqrt{n^2+m^2}}\right)\in \mathbb R^2\ \mid\ n,m\in \mathbb N\}$\\
        Решение: $A = \{\left(\frac{1}{\sqrt{1+m^2}},\frac{1}{\sqrt{n^2+1}}\right)\in \mathbb R^2\ \mid\ n,m\in \mathbb N\}$. Если зафиксируем $m$, то $\left(\frac{1}{\sqrt{1+m^2}},\frac{1}{\sqrt{n^2+1}}\right)$ стремится к $\left(\frac{1}{\sqrt{1+m^2}},0\right)$. $A$ является объединением таких последовательностей для всех $m \in \mathbb N$. Так как координаты точки симметричны, то $A'$ -- объединение $\{\left(0,\frac{1}{\sqrt{n^2+1}}\right)\in \mathbb R^2\ \mid\ n \in \mathbb N\}$ и $\{\left(\frac{1}{\sqrt{n^2+1}},0\right)\in \mathbb R^2\ \mid\ n \in \mathbb N\}$. Потому что такие предельные точки у $A$ точно есть, а вне их окрестностей лежит только конечное число элементов.
        \item (8) $A = \{(x,x^n)\in \mathbb R^2\ \mid\ n\in \mathbb N,\ x\in \mathbb R\}$\\
        Решение: $A$ -- объединение графиков функций $f(x) = x^n$. Так как $x \in \mathbb R$. Любая точка такого графика является предельной для $A$. ($x^n$ непрерывна). А вне этих графиков в $A$ нет точек, а значит и предельных тоже нет. Тогда $A' = A$.  
    \end{enumerate}
	
	\item (5) Пусть $A\subset\mathbb R$. Докажите, что существует такая последовательность $a_n\in A$, что $\lim\limits_{n\to +\infty} a_n = \sup A$.\\
	Решение: Если $\sup A$ лежит в $A$, то $a_n = \sup A$. Если не лежит, то $\forall \eps > 0\ \exists\ a\in A:|\sup A - a| < \eps$. Тогда возьмем такую последовательность $a_n = a$, где $|a - \sup A| < \frac{1}{n}$. Тогда с какого-то $n$ все $a_n$ будут попадать в $\eps$-окрестность $\sup A$. 
    \item Предъявите какую-нибудь функцию $N = N(\eps)$, для которой выполняется
    \[
        \forall \eps>0\ \forall n> N(\eps)\ |x_n-\lim\limits_{n\to+\infty} x_n|\leq \eps,
    \]
    если
    \begin{enumerate}
        \item (4) $x_n = \frac{n}{2^{\sqrt{n}}}$\\
        Решение: начиная с некоторого $n:\ n < 2^{\frac{\sqrt{n}}{2}}$. Тогда $\frac{n}{2^{\sqrt{n}}} < \frac{2^{\frac{\sqrt{n}}{2}}}{2^{\sqrt{n}}} = \frac{1}{2^{\frac{\sqrt{n}}{2}}}$ начиная с какого-то $n$. $2^{\frac{-\sqrt{n}}{2}} < \eps \Leftrightarrow \frac{-\sqrt{n}}{2} < \log_2\eps \Leftrightarrow \sqrt{n} > -2\log_2\eps$. $N(\eps) = \left \lceil{4\log^2_2\frac{1}{\eps}}\right \rceil + 1$. А чтобы $N$ было достаточно большим, пускай $N(\eps) = \max(10000,\ \left \lceil{4\log^2_2\frac{1}{\eps}}\right \rceil + 1)$
        \item (4) $x_n = n^{1/n}$\\
        Решение: предел равен 1. (Это докажется, если найду $N(\eps)$). $n^{\frac{1}{n}} > 1$, модуль раскрывается положительно. $n < (1 + \eps)^n, (1 + \eps)^n > 1 + n\eps,\ n < 1 + n\eps \Leftrightarrow n(1-\eps) < 1$. Если $\eps < 1$, то $N(\eps) = $ 
        \item (4) $x_n = \sqrt[3]{n^3 + n^2} - \sqrt[3]{n^3 - n^2}$\\
        Решение: предел равен $\frac{1}{3}$. $|\sqrt[3]{n^3 + n^2} - \sqrt[3]{n^3 - n^2} - \frac{1}{3}| < \eps \Leftrightarrow \frac{n^2}{(\sqrt[3]{n^3 + n^2})^2 + \sqrt[3]{n^6 - n^4}) + (\sqrt[3]{n^3 - n^2})^2}$
        \item (4) $x_n = \frac{(n-1)(n-2)\cdot\dots\cdot (n-10)}{(n+1)(n+2)\cdot\dots\cdot(n+10)}$
    \end{enumerate}
    
	\item При каких из нижеперечисленных условий числовая последовательность $x_n$ сходится?
	\begin{enumerate}
	    \item (3) $\exists a\in \mathbb R:\ \forall N>0\ \exists \eps>0:\ \forall n\geq\eps\ |x_n - a|<N$\\
        Решение: Если поменять $N$ и $\eps$, то получится определение предела. Сходится.
	    \item (3) $\forall \eps\in \mathbb R\ \exists N:\ \forall n>N\ \eps x_n < 1$\\
	    Решение: Если $\eps > 0$, поделим на $\eps$ и получим $x_n < \frac{1}{\eps}$. При положительных $x_n$ и больших $\eps$ это дает нам стремление к нулю. Если $\eps < 0$, то тоже поделим. $x_n > \frac{1}{\eps}$. И это дает стремление к нулю для отрицательных $x_n$ и большого по модулю $\eps$. Т.е. в итоге $x_n$ стремится к нулю.
	    \item (3) $\forall \eps>0\ \exists N:\ \forall n>N\ \left|\frac{x_1+x_2+\dots + x_n}{n}\right|\leq \eps$\\
	    Решение: $x_n = 1, 0, 1, 0, 0, 1, 0, 0, 0, 1, 0 \dots$ модуль среднего арифметического стремится к нулю, но сама она не сходится.
	    \item (3) $\forall \eps>0\ \exists N:\ \forall n>N\ x_n^2<\frac{1}{1+\eps}$\\
	    Решение: $\eps' = \frac{1}{1+\eps}$. $x_n^2 < \eps' \Leftrightarrow |x_n| < \sqrt{\eps'} \Rightarrow x_n$ сходится к 0, так как $\sqrt{\eps'}$ может быть любым положительным числом.
	    \item (3) $\forall\eps>0\ \exists N>0:\ \forall n>N\ 0<x_n^n\leq \frac{\eps}{n}$
	    \item (3) $\exists N:\ \forall n>0\ \frac{1}{n}\leq (x_n - 1)^n \leq n$
	\end{enumerate}
    
    \item Докажите, что последовательность $x_n$ сходится, и вычислите ее предел, если
    \begin{multicols}{2}
    \begin{enumerate}
        \item (7) $x_1=1,\quad x_{n+1}=\frac{x_n + 1}{x_n + 2}$
        \item (10) $x_1 = 2,\ x_2 = 1,\quad x_{n+2} = \sqrt[3]{x_{n+1}^2x_n}$
    \end{enumerate}
    \end{multicols}
    
    \item (10) При каких $x_1\in\mathbb{R}$ последовательность, заданная рекуррентно $x_{n+1}=x_n(2-x_n)$, $n\geqslant1$, сходится? Если сходится, найдите предел.
\end{enumerate}

\end{document}
















